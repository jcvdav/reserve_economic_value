\begin{table}[H]

\caption{\label{tab:w_fact}Invertebrate taxa, reference sizes, and weight factors used to convert invertebrate abundance into biomass}
\centering
\begin{tabular}[t]{>{\raggedright\arraybackslash}p{8em}>{\raggedright\arraybackslash}p{4em}>{\raggedleft\arraybackslash}p{4em}>{\raggedright\arraybackslash}p{17em}}
\toprule
Taxa & Reference size (mm) & Weight (gr / org) & Source\\
\midrule
\em{Holothuroidea} & - & 280.00 & Weight at first capture from \cite{ChavEz2011-eu}\\
\em{Strongylocentrotus purpuratus} & 45 & 36.87 & Minimum catch size (test diameter) from fisheries regulation \citep{Nom-007-sagpesc-20152015-yz}, allometric parameters estimated from data reported by \cite{Smith2021-ya}\\
\em{Mesocentrotus franciscanus} & 80 & 179.34 & Minimum catch size (test diameter) from fisheries regulation \citep{Nom-007-sagpesc-20152015-yz}, allometric parameters from \cite{Leus2013-ta}\\
\em{Haliotidae} & 136 & 330.32 & Minimum catch size from regulation \citep{Dof1987-oc}, allometric parameters from \cite{Rossetto2013-ba}\\
\em{Panulirus argus} & 223 & 388.12 & Minimum total length (equivalent to 135 mm abdominal length or 74.6 mm in cephalothoracic length) reported in the regulation \citep{Nom-006-sagpesc-20162016-ls}, allometric parameters from \cite{Neilson2011-yi}\\
\addlinespace
\em{Panulirus interruptus} & 175 & 395.50 & Minimum andominal length (equivalent 82.5 mm in cephalothoracic length) reported in the regulation \citep{Nom-006-sagpesc-20162016-ls}, allometric parameters from \cite{Murray1996-cf}\\
\bottomrule
\end{tabular}
\end{table}
